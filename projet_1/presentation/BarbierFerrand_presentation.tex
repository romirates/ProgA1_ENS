\begin{frame}{Tours de Hanoï}

Tour du Brahmâ, fin du monde et racine triangulaire.

\begin{enumerate}
\def\labelenumi{\arabic{enumi}.}
\tightlist
\item
  \textbf{Trois tours}
\end{enumerate}

Solution optimale : \(\Phi(N) = 2^{N}-1\)

\begin{enumerate}
\def\labelenumi{\arabic{enumi}.}
\setcounter{enumi}{1}
\tightlist
\item
  \textbf{Quatre tours}
\end{enumerate}

Solution optimale :
\(\Phi(N)=2^{\nabla 0}+2^{\nabla 1}+\cdots+2^{\nabla(N-1)}\) où
\(\nabla n\) est le plus grand entier \(p\) tel que
\(p(p+1)/2\leq n\).\footnote<.->{cf.
  \url{http://www.cnrs.fr/insmi/spip.php?article1170}}

\end{frame}

\begin{frame}{Un beau graphique}

\includegraphics{BarbierFerrand_presentation_files/figure-beamer/unnamed-chunk-1-1.pdf}

\end{frame}

\begin{frame}{Première implémentation}

\begin{enumerate}
\def\labelenumi{\arabic{enumi}.}
\tightlist
\item
  \textbf{Implémentation complète}
\end{enumerate}

\end{frame}

\begin{frame}{Deuxième implémentation}

\begin{enumerate}
\def\labelenumi{\arabic{enumi}.}
\tightlist
\item
  \textbf{Définition d'une nouvelle structure}\\
  \includegraphics{struct_donnee_hanoiv2.png}
\end{enumerate}

\end{frame}

\begin{frame}

\begin{enumerate}
\def\labelenumi{\arabic{enumi}.}
\setcounter{enumi}{1}
\tightlist
\item
  \textbf{Modification de la fonction mouvement}\\
  \includegraphics{fonction_mouvement.png}
\end{enumerate}

\end{frame}
